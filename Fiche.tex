\documentclass[a4paper]{article}
\usepackage[left=1.8cm,right=1.8cm,top=2cm,bottom=2cm]{geometry}

\usepackage[T1]{fontenc}
\usepackage[utf8]{inputenc}
\usepackage{etex,amsfonts,amssymb,amsmath,mathrsfs}
\usepackage{dsfont}
\usepackage{pifont}
\usepackage[tikz]{bclogo}
\usepackage{tikz,tkz-tab}
\usetikzlibrary{arrows,shadows,shapes,backgrounds,positioning,lindenmayersystems}
\usepackage{fancyhdr}
\usepackage{fancybox}
\setlength{\headheight}{12.81pt}
\usepackage[french]{babel}
\DecimalMathComma
\frenchbsetup{StandardLists=true}
\usepackage{lmodern}
\usepackage{xspace}
\usepackage[francais]{layout}
\usepackage{textcomp} %Pour simple quote droit \textquotesingle
\usepackage{fourier-orns}
\usepackage{colortbl}
\usepackage{multicol}
\usepackage{multirow}
\usepackage{numprint}
\usepackage{multido}
\usepackage{ifthen}
%%%%%%%%%%%%%%%%%%%%%%%%%%%%%%%%%%%%%%%%%%%%%%%%%%%%%%%%%%%%%%
\usepackage[french,lined,vlined,linesnumbered]{algorithm2e}
% \usepackage[french,lined,vlined,linesnumbered,boxed]{algorithm2e}
\setlength{\algomargin}{1em}
%%%%%%%%%%%%%%%%%%%%%%%%%%%%%%%%%%%%%%%%%%%%%%%%%%%%%%%%%%%%%%
\usepackage{stmaryrd}
\SetSymbolFont{stmry}{bold}{U}{stmry}{m}{n}
\usepackage{alltt}
\usepackage{moreverb}
\usepackage{appendix}

%%%%%%%%%%%%%%%%%%%%%%%%%%%%%%%%%%%%%%%%%%%%%%%%%%%%%%%%%%%%%%
%%%%%%%%%%%%%%%%%%%%%%%%%%%%%%%%%%%%%%%%%%%%%%%%%%%%%%%%%%%%%%

\usepackage{cellspace}
% Pour régler l'espacement vertical des filets et du texte dans un tableau
% \cellspacetoplimit=3pt
% \cellspacebottomlimit=3pt
% écrire Sc pour une colonne centrée

%%%%%%%%%%%%%%%%%%%%%%%%%%%%%%%%%%%%%%%%%%%%%%%%%%%%%%%%%%%%%%
%%%%%%%%%%%%%%%%%%%%%%%%%%%%%%%%%%%%%%%%%%%%%%%%%%%%%%%%%%%%%%

\graphicspath{{images/}}

%%%%%%%%%%%%%%%%%%%%%%%%%%%%%%%%%%%%%%%%%%%%%%%%%%%%%%%%%%%%%%
%%%%%%%%%%%%%%%%%%%%%%%%%%%%%%%%%%%%%%%%%%%%%%%%%%%%%%%%%%%%%%

% \renewcommand{\textbf}[1]{\begingroup\bfseries{\mathversion{bold}#1}\endgroup}

\usepackage{setspace}
\usepackage{amsmath}

\pagestyle{fancy}
\fancyfoot[C]{\thepage}
\fancyhead[L]{La Prépa des INP - Bordeaux}
\fancyhead[C]{}
\fancyhead[R]{Ondes Acoustiques dans les Fluides}
\renewcommand{\footrulewidth}{1pt}
\begin{document}
\centerline{\huge{\textbf{Fiche de Synthèse}}}
%\section*{Fiche de synthèse}
%\subsection*{\small{Ondes acoustiques dans les Fluides}}
\begin{center}
 \rule{1\linewidth}{1pt}
\end{center}


\section{Qu'est ce qu'une onde acoustique ?}
Une onde acoustique est une succession périodique de compressions et de détentes d'un fluide. Les grandeurs physiques caractéristiques sont donc la pression du fluide, sa vitesse et sa masse volumique.

\section{L'approximation acoustique}

\begin{itemize}
\item[$\bullet$] Une onde sonore en propagation dans un fluide peut être décrite en chaque point $M(x,y,z)$ et à chaque instant $t$ par les champs de pression (Pa), de masse volumique ($kg\cdot m^{-3}$) et de vitesse ($m\cdot s^{-1}$):

\begin{displaymath}
\left\lbrace
\begin{array}{ccl}
p_{tot}(x,y,z,t) & =  p_{0}+p(x,y,z,t) & avec  \left\langle p\right\rangle = 0 \\ 
\rho_{tot}(x,y,z,t) & =  \rho_{0}+\rho(x,y,z,t) & avec  \left\langle \rho\right\rangle = 0 \\
\overrightarrow{v}_{tot}(x,y,z,t) & = \overrightarrow{v}_{0}+\overrightarrow{v}(x,y,z,t) & avec \left\langle \overrightarrow{v}\right\rangle = 0
\end{array}\right.
\end{displaymath}

Les grandeurs indicées "$0$" sont les grandeurs au repos, supposées uniformes. 
$p$ est la \textbf{surpression} (ou pression acoustique) (en Pa)dû au passage de l'onde, et $\rho$ représente la variation de masse volumique.

\vspace{2mm}

\item[$\bullet$] \textbf{Approximation acoustique:} dans la description de l'onde sonore, on se limite à des \textbf{calculs à l'ordre 1}, ce qui permet d'écrire des équations linéaires. Dans le cadre de cette approximation, $\rho$ et $p$ sont donc des infiniment petits du premier ordre : 

\begin{displaymath}\begin{array}{ccc}
\left|p\right|\gg p_{0}, \left|\rho\right|\gg\rho_{0} & et & \left|\overrightarrow{v}\right|\ll c
\end{array}
\end{displaymath}
où $c$ est la célérité de l'onde acoustique.

\item[$\bullet$] Ce couplage avec l'approximation isentropique permet de déterminer l'équation de propagation des ondes lors de petits mouvement (hypothèse vérifier en pratique).

\item[$\bullet$] C'est une approximation de grandes longueurs d'ondes.
\end{itemize}


\section{Equation de propagation et forme des solutions.}

\section{Bilan énergétique}

\section{Coefficients de réflexion et de transmission à une interface sous incidence normale}
\begin{itemize}
\item Une onde sonore progressive plane en propagation dans le sens des $z$ croissants tombe sur une interface plane infinie située en $z=0$ entre deux fluides. L'onde est en partie réfléchie et en partie transmise. Sous incidence normale, à l'interface:
\item[-] Il y a \textbf{continuité des champs de vitesse}
\begin{center}
$\overrightarrow{v_{i}}(z=0,t)+\overrightarrow{v_{r}}(z=0,t)=\overrightarrow{v_{t}}(z=0,t)$
\end{center}
\item[-] Il y a \textbf{continuité des champs de surpression}
\begin{center}
$p_{i}(z=0,t)+p_{r}(z=0,t)=p_{t}(z=0,t)$
\end{center}

\item On considère la réflexion et la transmission d'une onde acoustique qui se propage dans un milieu $1$, et est incidente au niveau d'une interface. On considère les deux milieux ayant des impédances acoustiques différents dont on peut définir $2$ types de coefficients (associés à la propagation de la surpression ou de la vitesse). Les coefficients de réflexion $r_{v}$ et de transmission $t_{v}$ en amplitude des vitesses, les coefficients de réflexion $r_{p}$ et de transmission $t_{p}$ en amplitude des surpressions, sont à l'interface:
\begin{center}
\scalebox{1.2}{$r_{v}=\frac{Z_{1}-Z_{2}}{Z_{1}+Z_{2}}$} et \scalebox{1.2}{$t_{v}=\frac{2Z_{1}}{Z_{1}+Z_{2}}$}
\end{center}
\begin{center}
\scalebox{1.2}{$r_{p}=\frac{Z_{2}-Z_{1}}{Z_{1}+Z_{2}}$} et \scalebox{1.2}{$t_{p}=\frac{2Z_{2}}{Z_{1}+Z_{2}}$}
\end{center}

\item De même, les coefficients de réflexion R et de transmission T des puissances sonores sont :
\begin{center}
\scalebox{1.2}{$R=r_{v}^{2} =\left(\frac{Z_{1}-Z_{2}}{Z_{1}+Z_{2}}\right)^{2}$}
\end{center}
\begin{center}
\scalebox{1.2}{$T=t_{v}^{2}\frac{Z_{2}}{Z_{1}} =\frac{4Z_{1}Z_{2}}{(Z_{1}+Z_{2})^{\text{2}}}$}
\end{center}

\item On remarque que $R+T=1$, ce qui traduit \textbf{la conservation de l'énergie.}

\end{itemize}

\end{document}